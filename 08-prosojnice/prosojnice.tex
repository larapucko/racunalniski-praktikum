
\documentclass{beamer}

\usepackage{pgfpages}

{\theoremstyle{plain}
\newtheorem{izrek}{Izrek}}

{\theoremstyle{definition}
\newtheorem{envname}{caption}}

% Naloga 1.3.4: pripravite naslovno stran z vsebino:
% Več o tem, kako se naredi naslovno stran, si preberite na naslovu na naslovu:
% https://www.overleaf.com/learn/latex/Beamer
% To stran preberite do vključno razdelka "Creating a table of contents".
% Ukaz `\titlepage` deluje podobno kot ukaz `\maketitle`, ki ste ga že srečali.

\begin{document}

\title{Matematični izrazi in uporaba paketa \texttt{beamer}}
\subtitle{\emph{Matematičnih} nalog ni treba reševati!}
\institute{Fakulteta za matematiko in fiziko}
\date{}

\begin{frame}
	\titlepage
\end{frame}

\begin{frame}
    \frametitle{Kratek pregled}
    \tableofcontents
    % [pausesections]

\end{frame}


% Naloga 1.3.5: pripravite kazalo vsebine.
% 2. S pomožnim parametrom `pausesections' (v oglatih oklepajih) 
%    določite, da naj se kazalo vsebine odkriva postopoma.
%    Poglejte, kako deluje ta ukaz.
% 3. Ker ni videti preveč lepo, pomožni parameter zakomentirajte.

\section{Paket \texttt{beamer}}
\input{prosojnice/1-paket-beamer.tex}

\section{Paketa \texttt{amsmath} in \texttt{amsfonts}}

\section[Matematika, 1. del\\\large{Analiza, logika, množice}]{Matematika, 1. del}

\section{Stolpci in slike}

\section{Paket \texttt{beamer} in tabele}

\section[Matematika, 2. del\\\large{Zaporedja, algebra, grupe}]{Matematika, 2. del}

\end{document}